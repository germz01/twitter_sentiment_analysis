\documentclass[11pt]{article}

\usepackage[T1]{fontenc}
\usepackage[english]{babel}
\usepackage[hmarginratio=1:1,top=20mm,bottom=20mm,columnsep=20pt,left=15mm]{geometry}
\usepackage{multicol}
\usepackage[original]{abstract}
\usepackage{graphicx}
\usepackage{hyperref}
\usepackage{lipsum}

\renewcommand{\abstractnamefont}{\normalfont\bfseries}
\renewcommand{\abstracttextfont}{\normalfont\small}

\graphicspath{{images/}}

\title{\textbf{Human Language Technlogies \\ Twitter Sentiment Analysis}}
\author{Gianmarco Ricciarelli \\ \href{mailto:gianmarcoricciarelli@gmail.com}{gianmarcoricciarelli@gmail.com}}
\date{}

\begin{document}
    \maketitle

    \begin{abstract}
        \noindent
        Nowadays, when delving into the Social Network landscape, a user can choose among different
        paths, depending on the type of experience he/she is searching. Regardless the type of platform that is
        chosen by the user, either Facebook, or Instagram or Twitter, the amount of textual data that is
        produced every day is massive. With this paper, I describe the project I developed for the
        \textit{Human Language Technologies} class hosted by the University of Pisa's Master's Degree in
        Computer Science, that is, a prediction-oriented analysis of a dataset composed by more thant $40000$
        sentiment labeled tweets via a series of algorithms like SVM, CNN and LSTM.
    \end{abstract}

    \begin{multicols}{2}
        \section{Corpus Collection} % (fold)
        \label{sec:corpus_collection}
            \lipsum[1]
            \lipsum[1]
            \lipsum[1]
        % section corpus_collection (end)

        \section{Corpus Analysis} % (fold)
        \label{sec:corpus_analysis}

        % section corpus_analysis (end)
    \end{multicols}
\end{document}
